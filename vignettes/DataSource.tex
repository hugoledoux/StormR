% Options for packages loaded elsewhere
\PassOptionsToPackage{unicode}{hyperref}
\PassOptionsToPackage{hyphens}{url}
%
\documentclass[
]{article}
\usepackage{amsmath,amssymb}
\usepackage{lmodern}
\usepackage{iftex}
\ifPDFTeX
  \usepackage[T1]{fontenc}
  \usepackage[utf8]{inputenc}
  \usepackage{textcomp} % provide euro and other symbols
\else % if luatex or xetex
  \usepackage{unicode-math}
  \defaultfontfeatures{Scale=MatchLowercase}
  \defaultfontfeatures[\rmfamily]{Ligatures=TeX,Scale=1}
\fi
% Use upquote if available, for straight quotes in verbatim environments
\IfFileExists{upquote.sty}{\usepackage{upquote}}{}
\IfFileExists{microtype.sty}{% use microtype if available
  \usepackage[]{microtype}
  \UseMicrotypeSet[protrusion]{basicmath} % disable protrusion for tt fonts
}{}
\makeatletter
\@ifundefined{KOMAClassName}{% if non-KOMA class
  \IfFileExists{parskip.sty}{%
    \usepackage{parskip}
  }{% else
    \setlength{\parindent}{0pt}
    \setlength{\parskip}{6pt plus 2pt minus 1pt}}
}{% if KOMA class
  \KOMAoptions{parskip=half}}
\makeatother
\usepackage{xcolor}
\usepackage[margin=1in]{geometry}
\usepackage{color}
\usepackage{fancyvrb}
\newcommand{\VerbBar}{|}
\newcommand{\VERB}{\Verb[commandchars=\\\{\}]}
\DefineVerbatimEnvironment{Highlighting}{Verbatim}{commandchars=\\\{\}}
% Add ',fontsize=\small' for more characters per line
\usepackage{framed}
\definecolor{shadecolor}{RGB}{248,248,248}
\newenvironment{Shaded}{\begin{snugshade}}{\end{snugshade}}
\newcommand{\AlertTok}[1]{\textcolor[rgb]{0.94,0.16,0.16}{#1}}
\newcommand{\AnnotationTok}[1]{\textcolor[rgb]{0.56,0.35,0.01}{\textbf{\textit{#1}}}}
\newcommand{\AttributeTok}[1]{\textcolor[rgb]{0.77,0.63,0.00}{#1}}
\newcommand{\BaseNTok}[1]{\textcolor[rgb]{0.00,0.00,0.81}{#1}}
\newcommand{\BuiltInTok}[1]{#1}
\newcommand{\CharTok}[1]{\textcolor[rgb]{0.31,0.60,0.02}{#1}}
\newcommand{\CommentTok}[1]{\textcolor[rgb]{0.56,0.35,0.01}{\textit{#1}}}
\newcommand{\CommentVarTok}[1]{\textcolor[rgb]{0.56,0.35,0.01}{\textbf{\textit{#1}}}}
\newcommand{\ConstantTok}[1]{\textcolor[rgb]{0.00,0.00,0.00}{#1}}
\newcommand{\ControlFlowTok}[1]{\textcolor[rgb]{0.13,0.29,0.53}{\textbf{#1}}}
\newcommand{\DataTypeTok}[1]{\textcolor[rgb]{0.13,0.29,0.53}{#1}}
\newcommand{\DecValTok}[1]{\textcolor[rgb]{0.00,0.00,0.81}{#1}}
\newcommand{\DocumentationTok}[1]{\textcolor[rgb]{0.56,0.35,0.01}{\textbf{\textit{#1}}}}
\newcommand{\ErrorTok}[1]{\textcolor[rgb]{0.64,0.00,0.00}{\textbf{#1}}}
\newcommand{\ExtensionTok}[1]{#1}
\newcommand{\FloatTok}[1]{\textcolor[rgb]{0.00,0.00,0.81}{#1}}
\newcommand{\FunctionTok}[1]{\textcolor[rgb]{0.00,0.00,0.00}{#1}}
\newcommand{\ImportTok}[1]{#1}
\newcommand{\InformationTok}[1]{\textcolor[rgb]{0.56,0.35,0.01}{\textbf{\textit{#1}}}}
\newcommand{\KeywordTok}[1]{\textcolor[rgb]{0.13,0.29,0.53}{\textbf{#1}}}
\newcommand{\NormalTok}[1]{#1}
\newcommand{\OperatorTok}[1]{\textcolor[rgb]{0.81,0.36,0.00}{\textbf{#1}}}
\newcommand{\OtherTok}[1]{\textcolor[rgb]{0.56,0.35,0.01}{#1}}
\newcommand{\PreprocessorTok}[1]{\textcolor[rgb]{0.56,0.35,0.01}{\textit{#1}}}
\newcommand{\RegionMarkerTok}[1]{#1}
\newcommand{\SpecialCharTok}[1]{\textcolor[rgb]{0.00,0.00,0.00}{#1}}
\newcommand{\SpecialStringTok}[1]{\textcolor[rgb]{0.31,0.60,0.02}{#1}}
\newcommand{\StringTok}[1]{\textcolor[rgb]{0.31,0.60,0.02}{#1}}
\newcommand{\VariableTok}[1]{\textcolor[rgb]{0.00,0.00,0.00}{#1}}
\newcommand{\VerbatimStringTok}[1]{\textcolor[rgb]{0.31,0.60,0.02}{#1}}
\newcommand{\WarningTok}[1]{\textcolor[rgb]{0.56,0.35,0.01}{\textbf{\textit{#1}}}}
\usepackage{longtable,booktabs,array}
\usepackage{calc} % for calculating minipage widths
% Correct order of tables after \paragraph or \subparagraph
\usepackage{etoolbox}
\makeatletter
\patchcmd\longtable{\par}{\if@noskipsec\mbox{}\fi\par}{}{}
\makeatother
% Allow footnotes in longtable head/foot
\IfFileExists{footnotehyper.sty}{\usepackage{footnotehyper}}{\usepackage{footnote}}
\makesavenoteenv{longtable}
\usepackage{graphicx}
\makeatletter
\def\maxwidth{\ifdim\Gin@nat@width>\linewidth\linewidth\else\Gin@nat@width\fi}
\def\maxheight{\ifdim\Gin@nat@height>\textheight\textheight\else\Gin@nat@height\fi}
\makeatother
% Scale images if necessary, so that they will not overflow the page
% margins by default, and it is still possible to overwrite the defaults
% using explicit options in \includegraphics[width, height, ...]{}
\setkeys{Gin}{width=\maxwidth,height=\maxheight,keepaspectratio}
% Set default figure placement to htbp
\makeatletter
\def\fps@figure{htbp}
\makeatother
\setlength{\emergencystretch}{3em} % prevent overfull lines
\providecommand{\tightlist}{%
  \setlength{\itemsep}{0pt}\setlength{\parskip}{0pt}}
\setcounter{secnumdepth}{-\maxdimen} % remove section numbering
\ifLuaTeX
  \usepackage{selnolig}  % disable illegal ligatures
\fi
\IfFileExists{bookmark.sty}{\usepackage{bookmark}}{\usepackage{hyperref}}
\IfFileExists{xurl.sty}{\usepackage{xurl}}{} % add URL line breaks if available
\urlstyle{same} % disable monospaced font for URLs
\hypersetup{
  pdftitle={DataSource},
  hidelinks,
  pdfcreator={LaTeX via pandoc}}

\title{DataSource}
\author{}
\date{\vspace{-2.5em}2023-03-03}

\begin{document}
\maketitle

To run StormR functions users have to provide a tropical cyclone storm
track dataset in which the location and some characteristics of storms
are given across their lifespan.

\hypertarget{data-sources}{%
\subsection{Data sources}\label{data-sources}}

To run StormR functions users have to provide a tropical cyclone storm
track dataset in which the location and some characteristics of storms
are given across their lifespan. This dataset must be contained in a
netcdf file.We higlhy recommand using IBTrACS databases as they provide
the most extensive and relevant data overall. Nevertheless, other
databases can be used as long as the following fields are provided:

\begin{longtable}[]{@{}
  >{\centering\arraybackslash}p{(\columnwidth - 8\tabcolsep) * \real{0.1176}}
  >{\centering\arraybackslash}p{(\columnwidth - 8\tabcolsep) * \real{0.1765}}
  >{\centering\arraybackslash}p{(\columnwidth - 8\tabcolsep) * \real{0.3824}}
  >{\centering\arraybackslash}p{(\columnwidth - 8\tabcolsep) * \real{0.2059}}
  >{\centering\arraybackslash}p{(\columnwidth - 8\tabcolsep) * \real{0.1176}}@{}}
\toprule()
\begin{minipage}[b]{\linewidth}\centering
\textbf{Field name}
\end{minipage} & \begin{minipage}[b]{\linewidth}\centering
\textbf{Description}
\end{minipage} & \begin{minipage}[b]{\linewidth}\centering
\textbf{Format/Units}
\end{minipage} & \begin{minipage}[b]{\linewidth}\centering
\textbf{Example}
\end{minipage} & \begin{minipage}[b]{\linewidth}\centering
\textbf{Type}
\end{minipage} \\
\midrule()
\endhead
\texttt{basin} & Name of the area where the storm originated.
Traditionally divided into seven basins & IBTRaCS format \newline NA
(North Atlantic) \newline EP (Eastern North Pacific) \newline WP
(Western North Pacific) \newline NI (North Indian) \newline SI (South
Indian) \newline SP (Southern Pacific) \newline SA (South Atlantic) & SP
& Mandatory \\
\texttt{names} & Names of the storms & Capital letters & PAM &
Mandatory \\
\texttt{seasons} & Years of observations & Calendar year & 2015 &
Mandatory \\
\texttt{isoTime} & Date and time of observations & YYYY-MM-DD HH:mm:ss
UTC & 2015-03-13 12:00:00 & Mandatory \\
\texttt{lon} & Longitude of the observations & Eastern Decimal degrees &
168.7 & Mandatory \\
\texttt{lat} & Latitude of the observations & Northern Decimal degrees &
-17.6 & Mandatory \\
\texttt{msw} & Maximum sustained wind speed & \(m.s^{-1}\) & 77 &
Mandatory \\
\texttt{rmw} & Radius of maximum winds: distance between the center of
the storm and its band of strongest winds & \(km\) & 20 & Recommended \\
\texttt{pressure} & Central pressure & \(pa\) & 91100 & Recommended \\
\texttt{poci} & Pressure of the last closed isobar & \(pa\) & 92200 &
Recommended \\
\texttt{sshs} & Saffir-Simpson hurricane wind scale rating based on
\texttt{msw} (here in \(m.s^{-1}\)) & \(-1 =\) Tropical depression
(\texttt{msw} \(< 18\)) \newline \(0 =\) Tropical storm (\(18 \leq\)
\texttt{msw} \(< 33\)) \newline \(1 =\) Category 1 (\(33 \leq\)
\texttt{msw} \(< 42\)) \newline \(2 =\) Category 2 (\(42 \leq\)
\texttt{msw} \(< 49\)) \newline \(3 =\) Category 3 (\(49 \leq\)
\texttt{msw} \(< 58\)) \newline \(4 =\) Category 4 (\(58 \leq\)
\texttt{msw} \(< 70\)) \newline \(5 =\) Category 5 (\texttt{msw}
\(\ge 70\)) & 5 & Optional \\
\bottomrule()
\end{longtable}

\hypertarget{unit-conversion}{%
\subsubsection{Unit conversion}\label{unit-conversion}}

It is important to pay attention to the formats and units of data
provided by the database. If \texttt{msw}, \texttt{rmw},
\texttt{pressure} and/or \texttt{poci} fields units do not match the
required units (Cf Format/Units column), a conversion will have to be
performed during the extraction of data (See Define a StormsDataset
section).

\hypertarget{download-a-data-source}{%
\subsection{Download a Data Source}\label{download-a-data-source}}

\textbf{- Lien vers page téléchargement IBTraCKS OKKK}
\textbf{- Lien vers page téléchargement autre BDD ???}
\textbf{- Code pour download OKKK}

In the following example, we will use the data from the IBTrACS that
gather all storms that occured in the South Pacific Basin
(IBTrACS.SP.v04r00.nc). It is possible to download this file using:

\begin{Shaded}
\begin{Highlighting}[]
\FunctionTok{download.file}\NormalTok{(}\AttributeTok{url =} \StringTok{"https://www.ncei.noaa.gov/data/international{-}best{-}track{-}archive{-}for{-}climate{-}stewardship{-}ibtracs/v04r00/access/netcdf/IBTrACS.SP.v04r00.nc"}\NormalTok{,}
              \AttributeTok{destfile =} \StringTok{"./test\_ibtracs.nc"}\NormalTok{)}
\end{Highlighting}
\end{Shaded}

\hypertarget{define-a-stormsdataset}{%
\subsection{Define a StormsDataset}\label{define-a-stormsdataset}}

\textbf{- iniDataBase sur les 2 BDD téléchargées}
\textbf{- différence de gestion IBTRACKS car paramètres par défaut vs autre bdd ou il faut redéfinir les champs.}
\textbf{- dire qu'on a déjà une BDD par défaut de test déjà existant (e.g. ?IBTRACKS\_SP), pas besoin de faire un initDatabase OKKKK}
\textbf{- Gestion des unités ? OKKKKKK}

The \texttt{defDatabase} function let the user define a new dataset.
This function returns a \texttt{StormsDataset} object, especially
designed to store all the informations needed for the dataset (See Data
sources section).

In what follows, we define a new \texttt{StormsDataset} based on the
database we downloaded right above and the default settings of
\texttt{defDatabase} function.

\begin{Shaded}
\begin{Highlighting}[]
\NormalTok{sds }\OtherTok{\textless{}{-}} \FunctionTok{defDatabase}\NormalTok{(}\AttributeTok{filename =} \StringTok{"./test\_ibtracs.nc"}\NormalTok{,}
                   \AttributeTok{fields =} \FunctionTok{c}\NormalTok{(}\AttributeTok{basin =} \StringTok{"basin"}\NormalTok{,}
                              \AttributeTok{names =} \StringTok{"name"}\NormalTok{,}
                              \AttributeTok{seasons =} \StringTok{"season"}\NormalTok{,}
                              \AttributeTok{isoTime =} \StringTok{"iso\_time"}\NormalTok{,}
                              \AttributeTok{lon =} \StringTok{"usa\_lon"}\NormalTok{,}
                              \AttributeTok{lat =} \StringTok{"usa\_lat"}\NormalTok{,}
                              \AttributeTok{msw =} \StringTok{"usa\_wind"}\NormalTok{,}
                              \AttributeTok{sshs =} \StringTok{"usa\_sshs"}\NormalTok{,}
                              \AttributeTok{rmw =} \StringTok{"usa\_rmw"}\NormalTok{,}
                              \AttributeTok{pressure =} \StringTok{"usa\_pres"}\NormalTok{,}
                              \AttributeTok{poci =} \StringTok{"usa\_poci"}\NormalTok{),}
                   \AttributeTok{basin =} \StringTok{"SP"}\NormalTok{,}
                   \AttributeTok{seasons =} \FunctionTok{c}\NormalTok{(}\DecValTok{1980}\NormalTok{, }\FunctionTok{as.numeric}\NormalTok{(}\FunctionTok{format}\NormalTok{(}\FunctionTok{Sys.time}\NormalTok{(), }\StringTok{"\%Y"}\NormalTok{))),}
                   \AttributeTok{unit\_conversion =} \FunctionTok{c}\NormalTok{(}\AttributeTok{msw =} \StringTok{"knt\_to\_ms"}\NormalTok{,}
                                       \AttributeTok{rmw =} \StringTok{"nm\_to\_km"}\NormalTok{,}
                                       \AttributeTok{pressure =} \StringTok{"mb\_to\_pa"}\NormalTok{,}
                                       \AttributeTok{poci =} \StringTok{"mb\_to\_pa"}\NormalTok{),}
                   \AttributeTok{verbose =} \ConstantTok{TRUE}\NormalTok{)}
\end{Highlighting}
\end{Shaded}

Note: If one is extracting data from IBTrACS databases, we strongly
advise using the \texttt{usa\_} fields as they are the most filled and
relevant. The field \texttt{basin} here, will allow searching for storms
that occured only in the South Pacific Basin. The field \texttt{seasons}
here, will allow searching for storms that occured between 1980 and the
current year.

\hypertarget{unit-conversion-1}{%
\subsubsection{Unit conversion}\label{unit-conversion-1}}

In this example, IBRTrACS database provides maximum sustained wind
(\texttt{msw}) in knots, radius of maximum wind (\texttt{rmw}) in
nautical miles, pressure (\texttt{pressure}) and pressure at the
outermost closed isobar (\texttt{poci}) in millibar. The
\texttt{unit\_conversion} argument is then used to precise how to
convert units in the database. Here is a list of allowed arguments:

For \texttt{msw}: \newline \texttt{knt\_to\_ms}: converts knot in meter
per second \newline \texttt{kmh\_to\_ms}: converts kilometer per hour in
meter per second \newline \texttt{mph\_to\_ms}: converts miles per hour
in meter per second \newline \texttt{None}: No conversion needed
\newline

For \texttt{rmw}: \newline \texttt{nm\_to\_ms}: converts nautical miles
in kilometer \newline \texttt{None}: No conversion needed \newline

For both \texttt{pressure} and \texttt{poci}: \newline
\texttt{b\_to\_pa}: converts bar in pascal \newline \texttt{mb\_to\_pa}:
converts millibar in pascal \newline \texttt{atm\_to\_pa}: converts
atmosphere in pascal \newline \texttt{psi\_to\_pa}: converts psi in
pascal \newline \texttt{None}: No conversion needed \newline

These fields in \texttt{unit\_conversion} input are mandatory if they
are also contain in the fields of \texttt{fields} input. If a field does
not required any conversion, the user must still specify
\texttt{"None"}.

\hypertarget{dataset-example}{%
\subsubsection{Dataset example}\label{dataset-example}}

This package provides a dataset \texttt{test\_dataset} which gathers all
the tropical cyclones that occured nearby Vanuatu (resp. New Caledonia)
from 2015 to 2016 (resp. 2020 to 2021 ). It represents the default
\texttt{StormsDataset} object used in the \texttt{Storms} function if
non other \texttt{StormsDataset} object has been initialized with the
\texttt{defDatabase} function.

\hypertarget{visualisation-du-contenu-du-stormdataset}{%
\subsection{Visualisation du contenu du
StormDataset}\label{visualisation-du-contenu-du-stormdataset}}

\textbf{2-3 exemple :} \textbf{- Structure de l'objet}
\textbf{- liste des noims disponibles dans la BDD}

\begin{verbatim}
str(StormTestVanuatu)
unique(StormTestVanuatu@database$names))
\end{verbatim}

\hypertarget{reference}{%
\subsection{Reference}\label{reference}}

\begin{itemize}
\tightlist
\item
  Knapp, K. R., Kruk, M. C., Levinson, D. H., Diamond, H. J., \&
  Neumann, C. J. (2010). The International Best Track Archive for
  Climate Stewardship (IBTrACS). Bulletin of the American Meteorological
  Society, 91(3), Article 3.
  \url{https://doi.org/10.1175/2009bams2755.1}
\end{itemize}

\end{document}
