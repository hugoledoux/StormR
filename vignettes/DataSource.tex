% Options for packages loaded elsewhere
\PassOptionsToPackage{unicode}{hyperref}
\PassOptionsToPackage{hyphens}{url}
%
\documentclass[
]{article}
\usepackage{amsmath,amssymb}
\usepackage{lmodern}
\usepackage{iftex}
\ifPDFTeX
  \usepackage[T1]{fontenc}
  \usepackage[utf8]{inputenc}
  \usepackage{textcomp} % provide euro and other symbols
\else % if luatex or xetex
  \usepackage{unicode-math}
  \defaultfontfeatures{Scale=MatchLowercase}
  \defaultfontfeatures[\rmfamily]{Ligatures=TeX,Scale=1}
\fi
% Use upquote if available, for straight quotes in verbatim environments
\IfFileExists{upquote.sty}{\usepackage{upquote}}{}
\IfFileExists{microtype.sty}{% use microtype if available
  \usepackage[]{microtype}
  \UseMicrotypeSet[protrusion]{basicmath} % disable protrusion for tt fonts
}{}
\makeatletter
\@ifundefined{KOMAClassName}{% if non-KOMA class
  \IfFileExists{parskip.sty}{%
    \usepackage{parskip}
  }{% else
    \setlength{\parindent}{0pt}
    \setlength{\parskip}{6pt plus 2pt minus 1pt}}
}{% if KOMA class
  \KOMAoptions{parskip=half}}
\makeatother
\usepackage{xcolor}
\usepackage[margin=1in]{geometry}
\usepackage{color}
\usepackage{fancyvrb}
\newcommand{\VerbBar}{|}
\newcommand{\VERB}{\Verb[commandchars=\\\{\}]}
\DefineVerbatimEnvironment{Highlighting}{Verbatim}{commandchars=\\\{\}}
% Add ',fontsize=\small' for more characters per line
\usepackage{framed}
\definecolor{shadecolor}{RGB}{248,248,248}
\newenvironment{Shaded}{\begin{snugshade}}{\end{snugshade}}
\newcommand{\AlertTok}[1]{\textcolor[rgb]{0.94,0.16,0.16}{#1}}
\newcommand{\AnnotationTok}[1]{\textcolor[rgb]{0.56,0.35,0.01}{\textbf{\textit{#1}}}}
\newcommand{\AttributeTok}[1]{\textcolor[rgb]{0.77,0.63,0.00}{#1}}
\newcommand{\BaseNTok}[1]{\textcolor[rgb]{0.00,0.00,0.81}{#1}}
\newcommand{\BuiltInTok}[1]{#1}
\newcommand{\CharTok}[1]{\textcolor[rgb]{0.31,0.60,0.02}{#1}}
\newcommand{\CommentTok}[1]{\textcolor[rgb]{0.56,0.35,0.01}{\textit{#1}}}
\newcommand{\CommentVarTok}[1]{\textcolor[rgb]{0.56,0.35,0.01}{\textbf{\textit{#1}}}}
\newcommand{\ConstantTok}[1]{\textcolor[rgb]{0.00,0.00,0.00}{#1}}
\newcommand{\ControlFlowTok}[1]{\textcolor[rgb]{0.13,0.29,0.53}{\textbf{#1}}}
\newcommand{\DataTypeTok}[1]{\textcolor[rgb]{0.13,0.29,0.53}{#1}}
\newcommand{\DecValTok}[1]{\textcolor[rgb]{0.00,0.00,0.81}{#1}}
\newcommand{\DocumentationTok}[1]{\textcolor[rgb]{0.56,0.35,0.01}{\textbf{\textit{#1}}}}
\newcommand{\ErrorTok}[1]{\textcolor[rgb]{0.64,0.00,0.00}{\textbf{#1}}}
\newcommand{\ExtensionTok}[1]{#1}
\newcommand{\FloatTok}[1]{\textcolor[rgb]{0.00,0.00,0.81}{#1}}
\newcommand{\FunctionTok}[1]{\textcolor[rgb]{0.00,0.00,0.00}{#1}}
\newcommand{\ImportTok}[1]{#1}
\newcommand{\InformationTok}[1]{\textcolor[rgb]{0.56,0.35,0.01}{\textbf{\textit{#1}}}}
\newcommand{\KeywordTok}[1]{\textcolor[rgb]{0.13,0.29,0.53}{\textbf{#1}}}
\newcommand{\NormalTok}[1]{#1}
\newcommand{\OperatorTok}[1]{\textcolor[rgb]{0.81,0.36,0.00}{\textbf{#1}}}
\newcommand{\OtherTok}[1]{\textcolor[rgb]{0.56,0.35,0.01}{#1}}
\newcommand{\PreprocessorTok}[1]{\textcolor[rgb]{0.56,0.35,0.01}{\textit{#1}}}
\newcommand{\RegionMarkerTok}[1]{#1}
\newcommand{\SpecialCharTok}[1]{\textcolor[rgb]{0.00,0.00,0.00}{#1}}
\newcommand{\SpecialStringTok}[1]{\textcolor[rgb]{0.31,0.60,0.02}{#1}}
\newcommand{\StringTok}[1]{\textcolor[rgb]{0.31,0.60,0.02}{#1}}
\newcommand{\VariableTok}[1]{\textcolor[rgb]{0.00,0.00,0.00}{#1}}
\newcommand{\VerbatimStringTok}[1]{\textcolor[rgb]{0.31,0.60,0.02}{#1}}
\newcommand{\WarningTok}[1]{\textcolor[rgb]{0.56,0.35,0.01}{\textbf{\textit{#1}}}}
\usepackage{longtable,booktabs,array}
\usepackage{calc} % for calculating minipage widths
% Correct order of tables after \paragraph or \subparagraph
\usepackage{etoolbox}
\makeatletter
\patchcmd\longtable{\par}{\if@noskipsec\mbox{}\fi\par}{}{}
\makeatother
% Allow footnotes in longtable head/foot
\IfFileExists{footnotehyper.sty}{\usepackage{footnotehyper}}{\usepackage{footnote}}
\makesavenoteenv{longtable}
\usepackage{graphicx}
\makeatletter
\def\maxwidth{\ifdim\Gin@nat@width>\linewidth\linewidth\else\Gin@nat@width\fi}
\def\maxheight{\ifdim\Gin@nat@height>\textheight\textheight\else\Gin@nat@height\fi}
\makeatother
% Scale images if necessary, so that they will not overflow the page
% margins by default, and it is still possible to overwrite the defaults
% using explicit options in \includegraphics[width, height, ...]{}
\setkeys{Gin}{width=\maxwidth,height=\maxheight,keepaspectratio}
% Set default figure placement to htbp
\makeatletter
\def\fps@figure{htbp}
\makeatother
\setlength{\emergencystretch}{3em} % prevent overfull lines
\providecommand{\tightlist}{%
  \setlength{\itemsep}{0pt}\setlength{\parskip}{0pt}}
\setcounter{secnumdepth}{-\maxdimen} % remove section numbering
\ifLuaTeX
  \usepackage{selnolig}  % disable illegal ligatures
\fi
\IfFileExists{bookmark.sty}{\usepackage{bookmark}}{\usepackage{hyperref}}
\IfFileExists{xurl.sty}{\usepackage{xurl}}{} % add URL line breaks if available
\urlstyle{same} % disable monospaced font for URLs
\hypersetup{
  pdftitle={DataSource},
  hidelinks,
  pdfcreator={LaTeX via pandoc}}

\title{DataSource}
\author{}
\date{\vspace{-2.5em}2023-03-16}

\begin{document}
\maketitle

Before running \texttt{StormR} functions users have to provide a
tropical storm track dataset as a ``.nc'' (NetCDF) file in which the
location and some characteristics of storms are given across their
lifespan. This file is used to create a \texttt{StormsDataset} object
using the \texttt{defDatabase} function. By default, the arguments of
\texttt{defDatabase} function are set up to create a
\texttt{StormsDataset} object using the fields provided by USA agencies
in the IBTrACS database
\href{https://www.ncei.noaa.gov/products/international-best-track-archive}{International
Best Track Archive for Climate Stewardship} (Knapp et al., 2010). This
database provides a fairly comprehensive record of worldwide tropical
storms and cyclones with a 3-hours temporal resolution since 1841. Other
databases can be used as long as the following fields are provided:

\begin{longtable}[]{@{}
  >{\centering\arraybackslash}p{(\columnwidth - 8\tabcolsep) * \real{0.1176}}
  >{\centering\arraybackslash}p{(\columnwidth - 8\tabcolsep) * \real{0.1765}}
  >{\centering\arraybackslash}p{(\columnwidth - 8\tabcolsep) * \real{0.3824}}
  >{\centering\arraybackslash}p{(\columnwidth - 8\tabcolsep) * \real{0.2059}}
  >{\centering\arraybackslash}p{(\columnwidth - 8\tabcolsep) * \real{0.1176}}@{}}
\toprule()
\begin{minipage}[b]{\linewidth}\centering
\textbf{Field name}
\end{minipage} & \begin{minipage}[b]{\linewidth}\centering
\textbf{Description}
\end{minipage} & \begin{minipage}[b]{\linewidth}\centering
\textbf{Format/Units}
\end{minipage} & \begin{minipage}[b]{\linewidth}\centering
\textbf{Example}
\end{minipage} & \begin{minipage}[b]{\linewidth}\centering
\textbf{Type}
\end{minipage} \\
\midrule()
\endhead
\texttt{names} & Names of the storms & Capital letters & PAM &
Mandatory \\
\texttt{seasons} & Years of observations & Calendar year & 2015 &
Mandatory \\
\texttt{isoTime} & Date and time of observations & YYYY-MM-DD HH:mm:ss
UTC & 2015-03-13 12:00:00 & Mandatory \\
\texttt{lon} & Longitude of the observations & Eastern Decimal degrees &
168.7 & Mandatory \\
\texttt{lat} & Latitude of the observations & Northern Decimal degrees &
-17.6 & Mandatory \\
\texttt{msw} & Maximum sustained wind speed & \(m.s^{-1}\) & 77 &
Mandatory \\
\texttt{basin} & Name of the area where the storm originated.
Traditionally divided into seven basins & IBTRaCS format \newline NA
(North Atlantic) \newline EP (Eastern North Pacific) \newline WP
(Western North Pacific) \newline NI (North Indian) \newline SI (South
Indian) \newline SP (Southern Pacific) \newline SA (South Atlantic) & SP
& Recommended\(^{1}\) \\
\texttt{rmw} & Radius of maximum winds: distance between the center of
the storm and its band of strongest winds & \(km\) & 20 &
Recommended\(^{2}\) \\
\texttt{pressure} & Central pressure & \(pa\) & 91100 &
Recommended\(^{3}\) \\
\texttt{poci} & Pressure of the last closed isobar & \(pa\) & 92200 &
Recommended\(^{3}\) \\
\texttt{sshs} & Saffir-Simpson hurricane wind scale rating based on
\texttt{msw} (here in \(m.s^{-1}\)) & \(-1 =\) Tropical depression
(\texttt{msw} \(< 18\)) \newline \(0 =\) Tropical storm (\(18 \leq\)
\texttt{msw} \(< 33\)) \newline \(1 =\) Category 1 (\(33 \leq\)
\texttt{msw} \(< 42\)) \newline \(2 =\) Category 2 (\(42 \leq\)
\texttt{msw} \(< 49\)) \newline \(3 =\) Category 3 (\(49 \leq\)
\texttt{msw} \(< 58\)) \newline \(4 =\) Category 4 (\(58 \leq\)
\texttt{msw} \(< 70\)) \newline \(5 =\) Category 5 (\texttt{msw}
\(\ge 70\)) & 5 & Optional\(^{4}\) \\
\bottomrule()
\end{longtable}

\(^{1}\) This field let the user filter storms on a particular area. It
can be very usefull for large dataset if one is interested only in a
specific basin. If not provided, this filtering operation cannot be
achieved.

\(^{2}\) Providing \texttt{rmw} allows better estimates of wind speed
and direction. If not provided \texttt{rmw} is approximated using an
empirical formula derived from Willoughby et al.~(2006) :
\(R_m = 46.4e^{(-0.0155 \times v_m + 0.0169 \times |\phi|)}\).

\(^{3}\) If these fields are not filled, both Holland and Boose models
cannot be performed (See Behaviour vignette)

\(^{4}\) The \texttt{sshs} field is optional, if not provided this field
is automatically filled using the \texttt{msw} field.

\hypertarget{downloading-a-data-source}{%
\subsubsection{Downloading a data
source}\label{downloading-a-data-source}}

If not available, a storm track data set has to be downloaded. In the
following example we illustrate how a storm track data set can be
downloaded on the IBTrACKS website. Different ``.nc'' (NetCDF) files
containing storm track data sets can be downloaded here:
\url{https://www.ncei.noaa.gov/data/international-best-track-archive-for-climate-stewardship-ibtracs/v04r00/access/netcdf/}.
Please, follow the IBTrACS citation recommendations when using these
datasets (including citing Knapp et al., 2010 and Knapp et al., 2018,
and providing the access date for the later). For example, track data
for all the storms that occurred over the last three years over the
different basin world wide can be download on the working directory as
follows,

\begin{Shaded}
\begin{Highlighting}[]
\FunctionTok{download.file}\NormalTok{(}\AttributeTok{url =} \StringTok{"https://www.ncei.noaa.gov/data/international{-}best{-}track{-}archive{-}for{-}climate{-}stewardship{-}ibtracs/v04r00/access/netcdf/IBTrACS.last3years.v04r00.nc"}\NormalTok{, }\AttributeTok{destfile =} \StringTok{"./IBTrACS\_ALL\_3yrs.nc"}\NormalTok{)}
\end{Highlighting}
\end{Shaded}

\hypertarget{creating-a-stormsdataset}{%
\subsubsection{\texorpdfstring{Creating a
\texttt{StormsDataset}}{Creating a StormsDataset}}\label{creating-a-stormsdataset}}

Once a storm track data set is available a \texttt{StormsDataset} object
can be created using the \texttt{defDatabase} function as follows,

\begin{Shaded}
\begin{Highlighting}[]
\NormalTok{sds }\OtherTok{\textless{}{-}} \FunctionTok{defDatabase}\NormalTok{(}\AttributeTok{filename =} \StringTok{"IBTrACS\_ALL\_3yrs.nc"}\NormalTok{, }\AttributeTok{verbose =} \ConstantTok{FALSE}\NormalTok{)}
\FunctionTok{str}\NormalTok{(sds)}
\FunctionTok{length}\NormalTok{(sds}\SpecialCharTok{@}\NormalTok{database}\SpecialCharTok{$}\NormalTok{names)}
\end{Highlighting}
\end{Shaded}

This \texttt{StormsDataset} object contains track data for 344 storms.
When using an IBTrACKS storm track data set, default arguments can be
used. By default, field provided by USA agencies (starting with
``usa\_'') are used, but other fields can be selected. For example, for
the South Pacific basin (``SP''), we can choose to use data provided by
the Australian bureau of meteorology (starting with ``bom\_'') as
follows,

\begin{Shaded}
\begin{Highlighting}[]
\NormalTok{sds }\OtherTok{\textless{}{-}}\FunctionTok{defDatabase}\NormalTok{(}\AttributeTok{filename =} \StringTok{"IBTrACS\_ALL\_3yrs.nc"}\NormalTok{,}
                  \AttributeTok{fields =} \FunctionTok{c}\NormalTok{(}\AttributeTok{names =} \StringTok{"name"}\NormalTok{,}
                             \AttributeTok{seasons =} \StringTok{"season"}\NormalTok{,}
                             \AttributeTok{isoTime =} \StringTok{"iso\_time"}\NormalTok{,}
                             \AttributeTok{basin =} \StringTok{"basin"}\NormalTok{,}
                             \AttributeTok{lon =} \StringTok{"bom\_lon"}\NormalTok{,}
                             \AttributeTok{lat =} \StringTok{"bom\_lat"}\NormalTok{,}
                             \AttributeTok{msw =} \StringTok{"bom\_wind"}\NormalTok{,}
                             \AttributeTok{rmw =} \StringTok{"bom\_rmw"}\NormalTok{,}
                             \AttributeTok{pressure =} \StringTok{"bom\_pres"}\NormalTok{,}
                             \AttributeTok{poci =} \StringTok{"bom\_poci"}\NormalTok{),}
                  \AttributeTok{basin =} \StringTok{"SP"}\NormalTok{,}
                  \AttributeTok{verbose =} \ConstantTok{FALSE}\NormalTok{)}
\FunctionTok{length}\NormalTok{(sds}\SpecialCharTok{@}\NormalTok{database}\SpecialCharTok{$}\NormalTok{names)}
\end{Highlighting}
\end{Shaded}

This \texttt{StormsDataset} object contains track data for 38 storms in
the South Pacific basin.

\hypertarget{unit-conversion}{%
\subsubsection{Unit conversion}\label{unit-conversion}}

In the IBTrACS data sets speeds are given in knots (\(knt\)), distances
in nautical miles (\(nm\)), and pressure in millibar (\(mb\)) but the
models implemented in the \texttt{temporalBehaviour} and
\texttt{spatialBehaviour} functions to compute wind speed, wind
direction, and summary statistics require data following the
international system of units (SI) or SI-derived units. Therefore, by
default, when using IBTrACS data sets, the \texttt{defDatabase} function
converts knots (\(knt\)) to meter per second (\(m.s^{-1}\)), nautical
miles (\(nm\)) to kilometres (\(km\)), and millibar (\(mb\)) to Pascal
(\(pa\)). When using another source of data than IBTrACS, the
\texttt{unit\_conversion} argument can be used to convert the data to
the desire units.The following lists the allowed conversions available
when using the \texttt{unit\_conversion} argument

For \texttt{msw}: \newline \texttt{knt\_to\_ms}: converts knot in meter
per second (Default) \newline \texttt{kmh\_to\_ms}: converts kilometer
per hour in meter per second \newline \texttt{mph\_to\_ms}: converts
miles per hour in meter per second \newline \texttt{None}: No conversion
needed \newline

For \texttt{rmw}: \newline \texttt{nm\_to\_ms}: converts nautical miles
in kilometer (Default) \newline \texttt{None}: No conversion needed
\newline

For both \texttt{pressure} and \texttt{poci}: \newline
\texttt{b\_to\_pa}: converts bar in Pascal \newline \texttt{mb\_to\_pa}:
converts millibar in Pascal (Default) \newline \texttt{atm\_to\_pa}:
converts atmosphere in Pascal \newline \texttt{psi\_to\_pa}: converts
psi in Pascal \newline \texttt{None}: No conversion needed \newline

These fields in \texttt{unit\_conversion} argument are mandatory if they
are also contained in the fields of \texttt{fields} argument. If a field
does not required any conversion, the user must still specify
\texttt{"None"}.

\hypertarget{dataset-example}{%
\subsubsection{Dataset example}\label{dataset-example}}

This package provides a dataset \texttt{test\_dataset} which gathers all
the tropical cyclones that occured nearby Vanuatu (resp. New Caledonia)
from 2015 to 2016 (resp. 2020 to 2021 ). It represents the default
\texttt{StormsDataset} used in the \texttt{Storms} function if no other
\texttt{StormsDataset} has been initialized with the
\texttt{defDatabase} function.

\hypertarget{reference}{%
\subsection{Reference}\label{reference}}

\begin{itemize}
\item
  Knapp, K. R., Kruk, M. C., Levinson, D. H., Diamond, H. J., \&
  Neumann, C. J. (2010). The International Best Track Archive for
  Climate Stewardship (IBTrACS). Bulletin of the American Meteorological
  Society, 91(3), Article 3.
  \url{https://doi.org/10.1175/2009bams2755.1}
\item
  Knapp, K. R., H. J. Diamond, J. P. Kossin, M. C. Kruk, \& Schreck, C.
  J. (2018) International Best Track Archive for Climate Stewardship
  (IBTrACS) Project, Version 4. {[}indicate subset used{]}. NOAA
  National Centers for Environmental Information.
  \url{https://doi:10.25921/82ty-9e16} {[}2023-03-15{]}
\end{itemize}

\textbf{- Lien vers page téléchargement IBTraCKS OKKK}
\textbf{- Lien vers page téléchargement autre BDD ???}
\textbf{- Code pour download OKKK}

In the following example, we will use the data from the IBTrACS that
gather all storms that occurred in the South Pacific Basin
(IBTrACS.SP.v04r00.nc). It is possible to download this file using:

\begin{Shaded}
\begin{Highlighting}[]
\FunctionTok{download.file}\NormalTok{(}\AttributeTok{url =} \StringTok{"https://www.ncei.noaa.gov/data/international{-}best{-}track{-}archive{-}for{-}climate{-}stewardship{-}ibtracs/v04r00/access/netcdf/IBTrACS.SP.v04r00.nc"}\NormalTok{,}\AttributeTok{destfile=}\StringTok{"IBTrACS\_ALL\_3yrs.nc"}\NormalTok{)}
\end{Highlighting}
\end{Shaded}

\hypertarget{define-a-stormsdataset}{%
\subsection{Define a StormsDataset}\label{define-a-stormsdataset}}

\textbf{- iniDataBase sur les 2 BDD téléchargées}
\textbf{- différence de gestion IBTRACKS car paramètres par défaut vs autre bdd ou il faut redéfinir les champs.}
\textbf{- dire qu'on a déjà une BDD par défaut de test déjà existant (e.g. ?IBTRACKS\_SP), pas besoin de faire un initDatabase OKKKK}
\textbf{- Gestion des unités ? OKKKKKK}

The \texttt{defDatabase} function let the user define a new dataset.
This function returns a \texttt{StormsDataset} object, especially
designed to store all the informations needed for the dataset (See Data
sources section). In what follows, we define a new
\texttt{StormsDataset} based on the database we downloaded right above
and the default settings of \texttt{defDatabase} function.

\begin{Shaded}
\begin{Highlighting}[]
\NormalTok{sds }\OtherTok{\textless{}{-}} \FunctionTok{defDatabase}\NormalTok{(}\AttributeTok{filename =} \StringTok{"./test\_ibtracs.nc"}\NormalTok{,}
                   \AttributeTok{fields =} \FunctionTok{c}\NormalTok{(}\AttributeTok{basin =} \StringTok{"basin"}\NormalTok{,}
                              \AttributeTok{names =} \StringTok{"name"}\NormalTok{,}
                              \AttributeTok{seasons =} \StringTok{"season"}\NormalTok{,}
                              \AttributeTok{isoTime =} \StringTok{"iso\_time"}\NormalTok{,}
                              \AttributeTok{lon =} \StringTok{"usa\_lon"}\NormalTok{,}
                              \AttributeTok{lat =} \StringTok{"usa\_lat"}\NormalTok{,}
                              \AttributeTok{msw =} \StringTok{"usa\_wind"}\NormalTok{,}
                              \AttributeTok{sshs =} \StringTok{"usa\_sshs"}\NormalTok{,}
                              \AttributeTok{rmw =} \StringTok{"usa\_rmw"}\NormalTok{,}
                              \AttributeTok{pressure =} \StringTok{"usa\_pres"}\NormalTok{,}
                              \AttributeTok{poci =} \StringTok{"usa\_poci"}\NormalTok{),}
                   \AttributeTok{basin =} \StringTok{"SP"}\NormalTok{,}
                   \AttributeTok{seasons =} \FunctionTok{c}\NormalTok{(}\DecValTok{1980}\NormalTok{, }\FunctionTok{as.numeric}\NormalTok{(}\FunctionTok{format}\NormalTok{(}\FunctionTok{Sys.time}\NormalTok{(), }\StringTok{"\%Y"}\NormalTok{))),}
                   \AttributeTok{unit\_conversion =} \FunctionTok{c}\NormalTok{(}\AttributeTok{msw =} \StringTok{"knt\_to\_ms"}\NormalTok{,}
                                       \AttributeTok{rmw =} \StringTok{"nm\_to\_km"}\NormalTok{,}
                                       \AttributeTok{pressure =} \StringTok{"mb\_to\_pa"}\NormalTok{,}
                                       \AttributeTok{poci =} \StringTok{"mb\_to\_pa"}\NormalTok{),}
                   \AttributeTok{verbose =} \ConstantTok{TRUE}\NormalTok{)}
\end{Highlighting}
\end{Shaded}

Note: If one is extracting data from IBTrACS databases, we strongly
advise using the \texttt{usa\_} fields as they are the most filled and
relevant. The field \texttt{basin} here, will allow searching for storms
that occurred only in the South Pacific Basin. The field
\texttt{seasons} here, will allow searching for storms that occurred
between 1980 and the current year.

\end{document}
